\phantomsection\section*{ПРИЛОЖЕНИЕ А}\addcontentsline{toc}{section}{ПРИЛОЖЕНИЕ А}

\begin{lstlisting}[language=SQL,
	label=alg:perf_gen,
	caption={Функция генерации тестовых пользователей}]
CREATE EXTENSION IF NOT EXISTS "pgcrypto";
CREATE OR REPLACE FUNCTION generate_test_users(num_records INT)
RETURNS VOID AS $$
BEGIN
	INSERT INTO "User" (
		id, email, phone_number, password,
		first_name, last_name, gender, birth_date, role
	)
	SELECT 
		gen_random_uuid(),
		'user' || i || '@example.com',
		'+7' || lpad((i + 1000000000)::text, 10, '0'),
		'password' || i,
		initcap(substr(translate(md5(random()::text), '0123456789','abcdefghij'),1,8)),
		initcap(substr(translate(md5(random()::text), '0123456789','klmnopqrst'),1,10)),
		CASE WHEN random()>0.5 THEN 'мужской' ELSE 'женский' END,
		CURRENT_DATE - (365*20 + floor(random()*365*55))::int,
		'клиент'
	FROM generate_series(1, num_records) i;
END;
$$ LANGUAGE plpgsql;
\end{lstlisting}

\begin{lstlisting}[language=SQL,
	label=alg:perf_table,
	caption={Таблица для хранения результатов замеров времени}]
DROP TABLE IF EXISTS performance_metrics;
CREATE TABLE performance_metrics (
	table_size   INT,
	operation    TEXT,
	avg_time_ms  DOUBLE PRECISION
);
\end{lstlisting}

\begin{lstlisting}[language=SQL,
	label=alg:perf_study,
	caption={Исследование производительности операций \texttt{INSERT}, \texttt{SELECT}, \texttt{UPDATE} и \texttt{DELETE} на таблице \texttt{User} -- начало},
	captionpos=t,
	basicstyle=\ttfamily\small]
DO $$
DECLARE
	sizes int[] := ARRAY(SELECT generate_series(100, 100000, 1000));
	sz int;
	total_time double precision;
\end{lstlisting}

\begin{lstlisting}[language=SQL,
	label=alg:perf_stud3y,
	caption={Исследование производительности операций \texttt{INSERT}, \texttt{SELECT}, \texttt{UPDATE} и \texttt{DELETE} на таблице \texttt{User} -- продолжение},
	captionpos=t,
	basicstyle=\ttfamily\small]
	avg_time double precision;
	i int;
	u_id UUID;
	t1 timestamp;
	t2 timestamp;
BEGIN
	FOREACH sz IN ARRAY sizes LOOP
		RAISE NOTICE '========== % записей ========== ', sz;
		
		-- INSERT тест
		TRUNCATE TABLE "User" RESTART IDENTITY CASCADE;
		PERFORM generate_test_users(sz);
		total_time := 0;
		FOR i IN 1..10 LOOP
			u_id := gen_random_uuid();
			t1 := clock_timestamp();
			INSERT INTO "User"(
				id, email, phone_number, password, 
				first_name, last_name, gender, birth_date, role
			)
			VALUES (
				u_id, 
				'test'||i||'@example.com', 
				'+79001234567', 
				'pass', 
				'Имя', 
				'Фамилия', 
				'мужской', 
				CURRENT_DATE - INTERVAL '30 years', 
				'клиент'
			);
			t2 := clock_timestamp();
			DELETE FROM "User" WHERE id = u_id;
			total_time := total_time + EXTRACT(EPOCH FROM t2 - t1) * 1000;
		END LOOP;
		avg_time := total_time / 10;
		INSERT INTO performance_metrics VALUES (sz, 'INSERT', avg_time);
\end{lstlisting}

\begin{lstlisting}[language=SQL,
	label=alg:perf_stud1y,
	caption={Исследование производительности операций \texttt{INSERT}, \texttt{SELECT}, \texttt{UPDATE} и \texttt{DELETE} на таблице \texttt{User} -- продолжение},
	captionpos=t,
	basicstyle=\ttfamily\small]
		-- SELECT тест
		TRUNCATE TABLE "User" RESTART IDENTITY CASCADE;
		PERFORM generate_test_users(sz);
		total_time := 0;
		FOR i IN 1..10 LOOP
			t1 := clock_timestamp();
			SELECT id INTO u_id FROM "User" ORDER BY random() LIMIT 1;
			t2 := clock_timestamp();
			PERFORM 1 FROM "User" WHERE id = u_id;
			total_time := total_time + EXTRACT(EPOCH FROM t2 - t1) * 1000;
		END LOOP;
		avg_time := total_time / 10;
		INSERT INTO performance_metrics VALUES (sz, 'SELECT', avg_time);
		
		-- UPDATE тест
		TRUNCATE TABLE "User" RESTART IDENTITY CASCADE;
		PERFORM generate_test_users(sz);
		total_time := 0;
		FOR i IN 1..10 LOOP
			SELECT id INTO u_id FROM "User" ORDER BY random() LIMIT 1;
			t1 := clock_timestamp();
			UPDATE "User" SET first_name = 'Обновлено' WHERE id = u_id;
			t2 := clock_timestamp();
			total_time := total_time + EXTRACT(EPOCH FROM t2 - t1) * 1000;
		END LOOP;
		avg_time := total_time / 10;
		INSERT INTO performance_metrics VALUES (sz, 'UPDATE', avg_time);
\end{lstlisting}

\newpage
\begin{lstlisting}[language=SQL,
	label=alg:perf_study2,
	caption={Исследование производительности операций \texttt{INSERT}, \texttt{SELECT}, \texttt{UPDATE} и \texttt{DELETE} на таблице \texttt{User} -- конец},
	captionpos=t,
	basicstyle=\ttfamily\small]
		-- DELETE тест
		total_time := 0;
		FOR i IN 1..10 LOOP
			SELECT id INTO u_id FROM "User" ORDER BY random() LIMIT 1;
			t1 := clock_timestamp();
			DELETE FROM "User" WHERE id = u_id;
			t2 := clock_timestamp();
			INSERT INTO "User"(
				id, email, phone_number, password, 
				first_name, last_name, gender, birth_date, role
			)
			VALUES (
				gen_random_uuid(),
				'repl'||i||'@example.com', 
				'+79990000000', 
				'pass', 
				'Имя', 
				'Фамилия', 
				'женский', 
				CURRENT_DATE - INTERVAL '25 years', 
				'клиент'
			);
			total_time := total_time + EXTRACT(EPOCH FROM t2 - t1) * 1000;
		END LOOP;
		avg_time := total_time / 10;
		INSERT INTO performance_metrics VALUES (sz, 'DELETE', avg_time);
		TRUNCATE TABLE "User" RESTART IDENTITY CASCADE;
	END LOOP;
END $$;
\end{lstlisting}


