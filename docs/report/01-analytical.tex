\section{Аналитическая часть}

\subsection{Анализ предметной области}\label{scene}

\textbf{Фитнес-клуб} -- это учреждение, оснащённое оборудованием для физических упражнений, предоставляющие услуги в области физической активности и здоровья~\cite{IHRSA2019, leonquismondo2020service}. 

\textbf{Цель} фитнес-клуба -- обеспечение комфортных условий для физической активности клиентов, улучшении их здоровья и поддержании высокого уровня физической формы~\cite{IHRSA2019, leonquismondo2020service, bates2019health}.

\textbf{Основные функции} фитнес-клуба:
\begin{itemize}
	\item обеспечение тренировок и активного отдыха для клиентов;
	\item предоставление абонементов с различными условиями для доступа к услугам клуба;
	\item управление расписанием тренировок;
	\item учет посещений, тренировки и прогресса клиентов~\cite{bates2019health}.
\end{itemize}

\textbf{Клиенты} фитнес-клуба -- физические лица, которые заинтересованы в поддержании физической активности и улучшении здоровья~\cite{leonquismondo2020service, bates2019health}.

\textbf{Абонементы} являются основным способом доступа клиентов к услугам фитнес-клуба, и их типы определяются самим клубом в зависимости от потребностей и предпочтений клиентов. Например, могут быть предложены ежемесячные абонементы (предоставляющие доступ на один месяц) либо годовые абонементы (с выгодными условиями на длительный период)~\cite{bates2019health}.

Основными \textbf{сотрудниками} фитнес-клуба являются:
\begin{itemize}
	\item \textbf{администраторы}, которые управляют клиентской базой, занимаются продажей абонементов, отслеживанием посещаемости и предоставляют информацию о клубе.
	
	\item \textbf{тренеры} -- специалисты, которые проводят тренировки для клиентов~\cite{bates2019health}. 
\end{itemize}

\subsection{Обзор существующих решений}

\subsubsection*{1С:Фитнес клуб}

\textbf{1С:Фитнес клуб} -- это программное решение для автоматизации фитнес-клубов и спортивных учреждений, созданное компанией <<Лаборатория программного обеспечения>>. 

\begin{figure}[ht!]
	\begin{center}
		\includegraphics[scale=0.46]{./img/1c-1.png}
	\end{center}
	\caption{Интерфейс администратора для управления данными клиента фитнес-клуба приложения 1C:Фитнес клуб}
	\label{fig:1c-1}
\end{figure}

Приложение включает в себя функционал для работы с клиентской базой, маркетинга, расчета заработной платы, учета посещений, CRM-систему, интеграции с внешними сервисами и мобильные приложения для персонала. 

\begin{figure}[ht!]
	\begin{center}
		\includegraphics[scale=0.46]{./img/1c-2.png}
	\end{center}
	\caption{Интерфейс администратора для управления расписание тренировок фитнес-клуба приложения 1C:Фитнес клуб}
	\label{fig:1c-2}
\end{figure}

Приложение не предоставляет пробный период, но предлагает несколько тарифных планов, подходящих для разных масштабов бизнеса. Тарифы варьируются от 1 990 рублей в месяц за облачную подписку с базовым набором функций до 90 000 рублей за покупку программы с расширенными возможностями.

Для работы приложения необходима операционная система Windows. Для онлайн версии требуется постоянный интернет на компьютере.

\subsubsection*{fitness365}

\textbf{fitness365} -- это веб-система, специально разработанная для комплексного управления фитнес-клубом. Она предоставляет широкий набор инструментов, которые охватывают все аспекты работы клуба: от работы с клиентами до ведения статистики и аналитики.

fitness365 имеет интегрированную CRM-систему, которая позволяет клубу управлять базой клиентов -- каждому пользователю предоставляется персонализированный доступ.

\begin{figure}[ht!]
	\begin{center}
		\includegraphics[scale=0.35]{./img/f365-1.png}
	\end{center}
	\caption{Интерфейс главной страницы администратора фитнес-клуба приложения fitness365}
	\label{fig:f365-1}
\end{figure}

\begin{figure}[ht!]
	\begin{center}
		\includegraphics[scale=0.45]{./img/f365-2.png}
	\end{center}
	\caption{Интерфейс администратора для управления расписанием фитнес-клуба приложения fitness365}
	\label{fig:f365-2}
\end{figure}

Тарифы fitness365 предлагают различные опции для фитнес-клубов. Например, тариф <<Онлайн>> стоит 2 000 рублей в месяц, включает доступ через интернет и возможность работы с неограниченным числом клиентов, но поддерживает 5 рабочих мест. Для клубов, где могут быть проблемы с интернетом, доступна настольная версия за разовый платеж 44 000 рублей.

Для работы приложения необходима операционная система Windows, для онлайн версии требуется постоянный интернет на компьютере.

\subsubsection*{Сравнение решений}

\begin{table}[ht]
	\centering
	\begin{tabular}{|p{9cm}|c|c|}
		\hline
		\textbf{Функциональность для администратора} & \textbf{1С:Фитнес Клуб} & \textbf{fitness365} \\ \hline
		Управление расписанием, залами и персоналом & + & + \\ \hline
		Работа с абонементами и услугами & + & + \\ \hline
		Ведение клиентской базы и CRM & + & + \\ \hline
	\end{tabular}
	\begin{tabular}{|p{9cm}|c|c|}
		\hline
		\textbf{Функциональность для клиента} & \textbf{1С:Фитнес Клуб} & \textbf{fitness365} \\ \hline
		Наличие личного кабинета & + & + \\ \hline
		Покупка абонементов онлайн & + & + \\ \hline
		Самостоятельная запись на тренировки & + & + \\ \hline
	\end{tabular}
	\begin{tabular}{|p{9cm}|c|c|}
		\hline
		\textbf{Функциональность для тренера} & \textbf{1С:Фитнес Клуб} & \textbf{fitness365} \\ \hline
		Доступ к информации о клиентах & -- & + \\ \hline
		Просмотр и управление своим расписанием & -- & -- \\ \hline
		Мобильный доступ к системе & -- & -- \\ \hline
	\end{tabular}
	\caption{Сравнение функциональности по 1С:Фитнес Клуб и fitness365}
\end{table}

Рассматриваемые решения, такие как 1С:Фитнес Клуб и fitness365, могут быть избыточными для использования в условиях небольших или специализированных клубов. Кроме того, все рассматриваемые решения являются платными, что может стать препятствием для небольших фитнес-клубов. 

Разрабатываемая база данных будет сфокусирована для работы ключевых функций, таких как управление расписанием для тренеров и клиентов, самостоятельный контроль личного кабинета. Кроме того, программное обеспечение будет бесплатным, что делает его более доступным для широкого круга пользователей. 

\subsection{Формализация данных}

На рисунке~\ref{fig:er-chen} изображена диграмма <<сущность--связь>> в нотации Чена.

\begin{figure}[ht!]
	\begin{center}
		\includegraphics[scale=0.66]{./diag/er-chen.pdf}
	\end{center}
	\caption{Диаграмма <<сущность-связь>> фитнес-клуба в нотации Чена}
	\label{fig:er-chen}
\end{figure}

В базе данных для фитнес-клуба можно выделить следующие ключевые сущности.
\begin{enumerate}[label=\arabic*.]
	\item \textbf{Пользователь} -- основная сущность, представляющая всех участников системы: клиентов, тренеров и администраторов.
	
	\item \textbf{Тренер} -- сущность, представляющая пользователя, проводящего тренировки.
	
	\item \textbf{Специализация} -- сущность, которая определяет специализацию и используется для описания тренировок и компетенций тренеров.
	
	\item \textbf{Тип абонемента} -- сущность, описывающая варианты абонементов, устанавливаемые фитнес-клубом.
	
	\item \textbf{Абонемент} -- сущность, отражающая приобретённый пользователем абонемент.
	
	\item \textbf{Платеж} -- сущность, отражающая факт оплаты заказа.
	
	\item \textbf{Зал} -- сущность, которая содержит информацию о помещениях для проведения тренировок.
	
	\item \textbf{Тренировка} --  сущность, представляющая собой запланированное мероприятие -- тренировку.
	
	\item \textbf{Посещение} -- сущность, которая фиксирует факт участия клиента в конкретной тренировке.
\end{enumerate}

\subsection{Формализация пользователей и их прав доступа}

Взаимодействовать c базой данных будут четыре вида пользователей.

На рисунке~\ref{fig:use-case} приведена диаграмма вариантов использования базы данных.

\begin{figure}[ht!]
	\begin{center}
		\includegraphics[scale=0.70]{./diag/use-case.pdf}
	\end{center}
	\caption{Диаграмма вариантов использования базы данных}
	\label{fig:use-case}
\end{figure}

\textbf{Гость} -- анонимный пользователь, который может выполнять только определенные действия, такие как регистрация и вход в систему.

Авторизованные пользователи: 
\begin{enumerate}[label=---]
	\item \textbf{клиент} -- пользователь, который имеет минимальные права, ограниченные только возможностью взаимодействовать с данными, относящимися к его учетной записи и обслуживанию (например, заказами, абонементами и посещениями);
	
	\item \textbf{тренер} -- пользователь, имеющий права клиента, а также имеющий доступ к данным, связанным с его тренерской деятельностью;
	
	\item \textbf{администратор} -- пользователь, который имеет полный доступ к любым данным.
\end{enumerate}

\subsection{Модели данных}

\textbf{Модель данных} представляет собой формализованное описание структуры информационных единиц и операций с ними в информационной системе, которые определяет логическую организацию базы данных и способы хранения, организации и обработки данных~\cite[с. 4]{Avrunyev2018}.

Существует множество моделей данных, которые делятся на дореляционные(иерархические, сетевые, основнные на инвертированных списках), реляционные, постреляционные модели (например, ключ-значение, столбцовые, документные и графовые). Каждая из этих моделей имеет свои особенности и применяется в различных областях~\cite{Avrunyev2018}.

Основываясь на рейтинге из~\cite{DBEnginesRanking}, в дальнейшем будут рассмотрены наиболее популярные в настоящее время модели данных.

\subsubsection{Реляционная модель данных}

Реляционная модель была предложена Э. Коддом в 1970 году в статье~\cite{Kodd1970} и основана на теории отношений, опирается на математическое понятие $n$-арного отношения, что представляет собой подмножество декартового произведения.

Основными понятия реляционных баз данных являются тип данных, домен, атрибут, кортеж, отношение, первичный ключ.

\begin{figure}[ht!]
	\begin{center}
		\includegraphics[scale=0.7]{./img/rel-model.png}
	\end{center}
	\caption{Основные понятия реляционной модели данных~\cite[c. 31]{Avrunyev2018}}
	\label{fig:rel-model}
\end{figure}

\textbf{Атрибут} -- некоторая характеристика объекта (сущности), имеющая уникальное имя внутри отношения, через которое к ниму производится обращение.

\textbf{Кортеж} -- это совокупность значений, где каждое значение (или элемент) соответствует определенному атрибуту, и эти значения принадлежат соответствующему домену атрибута.

Множество всех кортежей образует \textbf{отношение}, которое является подмножеством декартового произведения доменов, причем количество кортежей в отношении называется \textit{мощностью отношения}.

\textbf{Схема отношения (заголовок отношения)} -- набор упорядоченных пар <A, T>, где A -- имя атрибута, а T -- домен. Количество атрибутов в схеме называется \textit{степенью отношения}.

Таким образом,  \textbf{реляционная база данных} -- это множество отношений, представленных в виде таблиц, где заголовок -- схема отношения, а строки -- кортежи.

Реляционная модель данных, по К. Дейту, состоит из трех частей.
 \begin{enumerate}[label=\arabic*.]
 		\item  \textbf{Структурная часть} -- описывает организацию данных, включая таблицы (отношения), атрибуты и их типы, а также связи между таблицами.
 		\item \textbf{Манипуляционная часть} -- включает операции над данными, такие как выборка, добавление, обновление и удаление данных, что осуществляется через язык запросов (например, SQL).
 		\item \textbf{Целостная часть} -- обеспечивает соблюдение целостности данных, включая ограничения (например, уникальность значений, ссылки между таблицами через внешние ключи), чтобы поддерживать корректность и непротиворечивость данных в базе~\cite[С. 30-35]{Avrunyev2018}.
 \end{enumerate}
 
 \subsubsection{Модель данных <<ключ-значение>>}
 
 \textbf{Модель данных ключ-значение} представляет собой структуру данных, которая использует ассоциативный массив, где каждый элемент состоит из \textit{уникального ключа} и \textit{связанного с ним значения}. Значение может быть любым типом данных, но оно не имеет структуры.
 
 Эта модель поддерживает \textbf{две основные операции}:
  \begin{enumerate}[label=---]
  	\item \textbf{получение значения по ключу} -- если ключ существует, возвращается связанное с ним значение, иначе -- NULL (специальное значение, которое используется в базах данных для обозначения отсутствия данных или неизвестного значения).
  	\item \textbf{запись значения по ключу} -- позволяет добавить или обновить значение для определенного ключа (также возможно установить время жизни ключа, после чего он будет автоматически удален) ~\cite[С. 89-91]{Avrunyev2018}.
  \end{enumerate}

 
\subsubsection{Документная модель данных}

Модель документная расширяет представление модели <<ключ-значение>>, позволяя хранить более сложные структуры данных -- документы.

\textbf{Документная модель данных} -- это подход к организации и хранению данных, при котором данные представлены в виде документов.

\textbf{Основные характеристики документной модели}:
 \begin{enumerate}[label=\arabic*)]
	\item документ как основная единица хранения;
	\item \textbf{документ} -- это структурированный набор данных, который может содержать различные пары \textit{ключ-значение} и может включать
	\begin{enumerate}[label=---]
		\item простые типы данных: строки, числа, булевы значения;
		\item упорядоченные списки значений;
		\item вложенные документы: другие объекты, которые могут быть представлены в формате пары ключ-значение;
		\item сложные типы данных: например, даты, бинарные данные и другие;
	\end{enumerate}
	\item отсутствие операций соединения -- cвязанные данные обычно хранятся в одном документе~\cite[С. 92-96]{Avrunyev2018}.
\end{enumerate}

\subsubsection*{Выбор модели данных}

В таблице~\ref{dif} представлено сравнение трёх моделей данных -- реляционной модели (РМ), модели ключ–значение (КЗМ) и документной модели (ДМ). Для удобства использованы условные обозначения: <<+>> -- высокая степень поддержки критерия, <<->> — низкая или отсутствующая поддержка, <<+/->> — частичная или ограниченная поддержка.

\begin{table}[h!]
	\centering
	\begin{tabular}{|p{5.2cm}|>{\centering\arraybackslash}p{2.3cm}|>{\centering\arraybackslash}p{2.3cm}|>{\centering\arraybackslash}p{2.3cm}|}
		\hline
		\textbf{Критерий} & \textbf{РМ} & \textbf{КЗМ} & \textbf{ДМ} \\
		\hline
		Поддержка связей между сущностями & + & - & +/- \\
		\hline
		Структурированность данных & + & - & +/- \\
		\hline
		Масштабируемость & +/- & + & + \\
		\hline
	\end{tabular}
	\caption{Сравнение моделей данных для базы данных фитнес-клуба}
	\label{dif}
\end{table}

Для базы данных фитнес-клуба выбрана реляционная модель ввиду её строгой структурированности и поддержки связей между сущностями.

Для кэширования выбрана модель ключ–значение, обеспечивающая гибкость и масштабируемость при работе с простыми данными, не требующими сложной структуры и поддержки связей, что делает документную модель избыточной для данной задачи.

\subsection{Формализация задачи}

Предметная область охватывает деятельность фитнес-клуба, включая управление данными о клиентах (их личной информации, тренировках, истории посещений), тренерах (их расписаниях, специализациях) и организации тренировок, расписаний, а также систему учета абонементов. 

Цель создания базы данных -- автоматизация учёта и повышение эффективности работы фитнес-клуба и качества обслуживания клиентов.

Для взаимодействия с базой данных фитнес-клуба необходимо разработать интерфейс, который обеспечит функциональный доступ к ключевым операциям системы в зависимости от роли пользователя.

Интерфейс должен предусматривать следующие возможности:
\begin{itemize}
	\item регистрация и авторизация пользователей;
	\item приобретение абонементов;
	\item запись на тренировки и управление расписанием.
\end{itemize}

В рамках данной курсовой работы не рассматривается вопрос обеспечения конфиденциальности персональных данных пользователей и интеграции платежной системы:
\begin{itemize}
	\item все данные, включая контактные и учетные данные пользователей, будут храниться в базе данных без использования специализированных механизмов защиты конфиденциальности;
	\item не реализуется функциональность для работы с платежными системами и интеграция с внешними платёжными сервисами для обработки финансовых транзакций или хранения данных о платежах.
\end{itemize}

\subsection*{Вывод}

В данном разделе проведен анализ предметной области фитнес-клубов, описана структура базы данных, рассмотрены пользователи базы данных и их права доступа. Также проведен анализ различных моделей данных, в результате чего для хранения данных выбрана реляционная модель данных, а для кэширования -- модель <<ключ-значение>>. Рассмотрены существующие решения и их особенности.
