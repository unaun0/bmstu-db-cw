\phantomsection\section*{ВВЕДЕНИЕ}\addcontentsline{toc}{section}{ВВЕДЕНИЕ}

В последние годы наблюдается значительный рост интереса к физической активности и занятиям спортом, что, в свою очередь, привело к увеличению числа услуг, ориентированных на благополучие, здоровье и активный образ жизни, стремящихся удовлетворить растущий спрос и потребности общества~\cite{IHRSA2019}. 

Современные фитнес-клубы представляют собой не только места для занятий спортом, но и полноценные бизнес-структуры, в которых осуществляется широкий спектр процессов: регистрация и обслуживание клиентов, планирование и проведение тренировок, управление персоналом и тренерами, ведение финансовой отчетности и многое другое. В связи с этим возникает необходимость в создании надежной и гибкой информационной системы, основой которой является качественно спроектированная база данных.

\textbf{Цель работы} -- разработать базу данных для хранения и обработки данных фитнес-клуба. 

Для достижения цели курсовой работы необходимо выполнить следующие \textbf{задачи}: 

\begin{enumerate}[label=\arabic*)]
	\item провести анализ предметной области и формализовать задачу;
	\item описать структуру базы данных и ее пользователей;
	\item провести анализ моделей данных и выбрать наиболее подходящую;
	\item спроектировать требуемую базу данных;
	\item спроектировать триггеры для автоматического обновления данных;
	\item выбрать средства реализации базы данных;
	\item реализовать спроектированную базу данных и обеспечить её функциональность согласно проектным решениям;
	\item реализовать интерфейс доступа к базе данных;
	\item провести исследование производительности базы данных при увеличения объема данных, количества одновременных запросов, а также с использованием кеширование и без него.
\end{enumerate}

