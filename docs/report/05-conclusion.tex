\phantomsection\section*{ЗАКЛЮЧЕНИЕ}\addcontentsline{toc}{section}{ЗАКЛЮЧЕНИЕ}

Цель курсовой работы, заключавшаяся в разработке базы данных для хранения и обработки данных фитнес-клуба, была успешно достигнута. В ходе работы были выполнены все поставленные задачи.

Был проведен анализ предметной области и формализована задача, что позволило определить требования к базе данных. Разработана структура базы данных с определением ролей пользователей системы. После анализа различных моделей данных была выбрана наиболее оптимальная модель для данного проекта.

В процессе проектирования базы данных были спроектированы необходимые сущности и их взаимосвязи, а также ролевая модель для разграничения прав доступа. Также были разработаны триггеры для автоматического обновления данных и внедрены ограничения целостности данных, что обеспечило корректность хранимой информации. Все проектные решения были успешно реализованы, а триггеры протестированы.

Для взаимодействия с базой данных был реализован интерфейс.

Проведенные исследования производительности базы данных показали, что кеширование значительно улучшает время отклика по сравнению с прямыми запросами к базе данных. Также было установлено, что с увеличением объема данных и количества одновременных запросов время отклика системы увеличивается, что указывает на рост нагрузки на базу данных.